\documentclass[12pt,a4paper]{book}
\usepackage{fontspec,titlesec,fancyhdr,ragged2e,blindtext,indentfirst,graphicx,xunicode,xltxtra,xgreek,lipsum}
% xgreek package required for greek hypernation
\usepackage{showframe}
\usepackage{layout}
% \usepackage{}
\pagestyle{fancy}
\lhead{}
\chead{}%\parbox[l]{500cm}}
\rhead{}
\lfoot{}
\cfoot{}
% \fancyhf{}
%\thepage
%\rfoot{\rightskip=-0.8in \thepage}
\fancyfoot[RO]{\rightskip=-1.67in \thepage}
\fancyfoot[RE]{\leftskip=5.32in \thepage}
\renewcommand\headrule{}
\usepackage[margin=0.7in,footskip=0.35in]{geometry}
\textwidth = 510pt %right page margin
%\setmainlanguage[variant=mono]{greek}
\newcommand{\en}[1]{\setlanguage{american}#1\setlanguage{moderngreek}}
\setmainfont{Arial}
% \makeatletter
% \newcommand\thefontsize[1]{{#1 The current font size is: \f@size pt\par}}
% \makeatother
\begin{document}
interwordspace: \the\fontdimen2\font \    
interwordstretch: \the\fontdimen3\font \    
emergencystretch: \the\emergencystretch\par
\blindtext
\layout{}
%\section{Font Sizes}
% \sffamily
% \thefontsize\tiny
% \thefontsize\scriptsize
% \thefontsize\footnotesize
% \thefontsize\small
% \thefontsize\normalsize
% \thefontsize\large
% \thefontsize\Large
% \thefontsize\LARGE
% \thefontsize\huge
% \thefontsize\Huge
\section*{\centerline{Ο χουλιγκανισμός στα
σχολεία}}\subsection*{\centerline{Ορισμός και ετυμολογία}}
\noindent\setlength{\parindent}{20pt}\indent Ως χουλιγκανισμός αρχικά
αναφέρεται η ανάρμοστη, βίαιη και αντικοινωνική συμπεριφορά οπαδών αθλητικών
ομάδων, που οδηγεί στη διατάραξη της τάξης και συνδέεται με βαθύτερα κοινωνικά
αίτια, έχοντας αντίκτυπο σε όλες τις δομές της κοινωνικοπολιτικής ζωής της
χώρας. Η ρατσιστική αυτή διάθεση και οι ομαδικές συγκρούσεις οπαδών εντός και 
εκτός γηπέδων καταλήγουν πολλές φορές σε δολοφονίες, οι οποίες ουσιαστικά τελούνται χωρίς προφανή λόγο και ανεξάρτητα από αθλητικούς αγώνες, με τη βία να εξαπλώνεται επικίνδυνα ολοένα και περισσότερο στις χώρες τις Ευρώπης, όπως στην Ιταλία, Ισπανία, Τουρκία, Ολλανδία, Νορβηγία, Πολωνία, Κροατία, Ρωσία και Ελλάδα, ιδίως μετά το 1970.
\newline\setlength{\parindent}{20pt}\indent Ιστορικά Ο όρος χουλιγκανισμός
(``hooliganism``) άρχισε να χρησιμοποιείται απο τη δεκαετία του 1890, για να
χαρακτηρίσει τη συμπεριφορά συμμοριών των δρόμων στο Λονδίνο και έλκει την ονομασία του απο τον Ιρλανδό Πάτρικ Χούλιγκαν (Patrick Hooligan
), ο οποίος υπήρξε διαβόητος βάνδαλος και εγκληματίας στο Λονδίνο το 
1898.\newline 
Εκτός από την Ευρώπη, φαινόμενα χουλιγκανισμού έχουν παρατηρηθεί και στην
Αμερική, με τον όρο  να γίνεται διαχρονικός, εκφράζοντας και στις μέρες μας τον
``τρόπο ζωής`` κάποιων ανεγκέφαλων κλειστών, αλλά πολυπληθών ομάδων ανθρώπων, οι
οποίοι δρούν βίαια - σύμφωνα με την ψυχολογία της μάζας - κατά τον Φρόυντ, η
οποία (μάζα) λειτουργεί σύμφωνα με τον επηρεασμό και τη χειραγώγησή της, όπως υποστηρίζει ο γάλλος μαίτρ της κοινωνικής ψυχολογίας Γκουστάβ Λε Μπόν.
\subsection*{Αίτια και μορφές του μαθητικού χουλιγκανισμού}
\setlength{\parindent}{20pt}\indent Οι εκπαιδευτικές κοινότητες
Οι εκπαιδευτικές κοινότητες, ως μικρόκοσμος της κοινωνίας, δυστυχώς παρουσιάζουν
όλα τα χαρακτηριστικά και τις παθογένειες της οικογενειακής και κοινωνικής
δομής, από την οποία προέρχονται. Η απότομη μετάβαση από την αυταρχική στην
αντιαυταρχική εκπαίδευση των τελευταίων τριανταπέντε ετών, χωρίς να έχει περάσει
ένα στάδιο προετοιμασίας, έκανε το μαθητή από  τρομοκρατούμενο, τρομοκράτη. Η
επικράτηση ανεπαρκών εγχειριδίων μάθησης και η διδακτική ``ασυδοσία`` από
ανεκπαίδευτους ουσιαστικά ``δασκάλους`` και λανθασμένα εκπαιδευτικά συστήματα,
που επεβλήθησαν από το Υπουργείο Παιδείας των εκάστοτε Κυβερνήσεων, έκαναν τους
μαθητές όλων των βαθμίδων εκπαίδευσης ``πειραματόζωα``. Γι' αυτό και τα
κρούσματα πνευματικού χουλιγκανισμού, δηλαδή διανοητικής αλητείας, έχουν
επικίνδυνα πολλαπλασιασθεί στα σχολεία της Μέσης ιδιαίτερα Εκπαίδευσης, αφού το
μοντέλο της Παγκοσμοιοποίησης, θέλει παιδιά ρομπότ - εξαρτήματα και υπηρέτες του
συστήματος και όχι υγιώς σκεπτόμενους ανθρώπους.  
\end{document}