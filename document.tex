\documentclass[12pt,a4paper]{book}
\usepackage{fontspec,titlesec,fancyhdr}
% \usepackage{}
\pagestyle{fancy}
\lhead{}
\chead{}
\rhead{}
\lfoot{}
\cfoot{}
% \fancyhf{}
\rfoot{\thepage}
\renewcommand\headrule{}
\setmainfont{Arial}
% \makeatletter
% \newcommand\thefontsize[1]{{#1 The current font size is: \f@size pt\par}}
% \makeatother
\begin{document}
%\section{Font Sizes}
% \sffamily
% \thefontsize\tiny
% \thefontsize\scriptsize
% \thefontsize\footnotesize
% \thefontsize\small
% \thefontsize\normalsize
% \thefontsize\large
% \thefontsize\Large
% \thefontsize\LARGE
% \thefontsize\huge
% \thefontsize\Huge
\section*{Ο χουλιγκανισμός στα σχολεία}
\subsection*{Ορισμός και ετυμολογία}
\makeatletter\@openrightfalse
Σε περίπτωση που ο compiler το κρίνει απαραίτητο, μπορεί
να κόψει με κατάλληλο τρόπο τις λέξεις σύμφωνα με τη γραμματική της ελληνικής γλώσσας. Ως τέτοιο παράδειγμα μπορούμε να δούμε στο pdf αρχείο τη λέξη: αεροπλάνο.
Από το επόμενο παράδειγμα και μετά, θα είσαστε σε θέση να ελέγχετε σε μεγάλο
βαθμό το αποτέλεσμα που ζητάτε ως προς την εμφάνιση του κειμένου σας.
Μέχρι τότε, καλό θα ήταν να δείτε το αποτέλεσμα αν πχ βάλετε το σύμβολο του σχολίου πριν από την
εντολή date και κάνετε compile. Επίσης θα μπορούσατε να δείτε τη διαφορά, όταν βάλετε το σύμβολο του σχολίου πριν από την
εντολή maketitle. Τέλος, μπορείτε να δοκιμάσετε να συμπληρώσετε όνομα στο author, τίτλο στο title
και ημερομηνία στο date (χωρίς να έχετε την εντολή maketitle σε σχόλιο...)
\end{document}