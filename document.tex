\documentclass[12pt,a4paper]{book}
\usepackage{fontspec,titlesec,fancyhdr,ragged2e,blindtext,indentfirst,graphicx,xunicode,xltxtra,xgreek,lipsum}
% xgreek package required for greek hypernation
\usepackage{showframe}
\usepackage{layout}
% \usepackage{}
\pagestyle{fancy}
\lhead{}
\chead{}%\parbox[l]{500cm}}
\rhead{}
\lfoot{}
\cfoot{}
% \fancyhf{}
%\thepage
%\rfoot{\rightskip=-0.8in \thepage}
\fancyfoot[RO]{\rightskip=-1.67in \thepage}
\fancyfoot[RE]{\leftskip=5.32in \thepage}
\renewcommand\headrule{}
\usepackage[margin=0.7in,footskip=0.35in]{geometry}
\textwidth = 510pt %right page margin
%\setmainlanguage[variant=mono]{greek}
\newcommand{\en}[1]{\setlanguage{american}#1\setlanguage{moderngreek}}
\setmainfont{Arial}
% \makeatletter
% \newcommand\thefontsize[1]{{#1 The current font size is: \f@size pt\par}}
% \makeatother

\begin{document}
interwordspace: \the\fontdimen2\font \    
interwordstretch: \the\fontdimen3\font \    
emergencystretch: \the\emergencystretch\par
\blindtext
\layout{}
%\section{Font Sizes}
% \sffamily
% \thefontsize\tiny
% \thefontsize\scriptsize
% \thefontsize\footnotesize
% \thefontsize\small
% \thefontsize\normalsize
% \thefontsize\large
% \thefontsize\Large
% \thefontsize\LARGE
% \thefontsize\huge
% \thefontsize\Huge
\section*{\centerline{Ο χουλιγκανισμός στα
σχολεία}}\subsection*{\centerline{Ορισμός και ετυμολογία του όρου}}
\noindent\setlength{\parindent}{20pt}\indent Ως χουλιγκανισμός αρχικά
αναφέρεται η ανάρμοστη, βίαιη και αντικοινωνική συμπεριφορά οπαδών αθλητικών
ομάδων, που οδηγεί στη διατάραξη της τάξης και συνδέεται με βαθύτερα κοινωνικά
αίτια, έχοντας αντίκτυπο σε όλες τις δομές της κοινωνικοπολιτικής ζωής της
χώρας. Η ρατσιστική αυτή διάθεση και οι ομαδικές συγκρούσεις οπαδών εντός και 
εκτός γηπέδων καταλήγουν πολλές φορές σε δολοφονίες, οι οποίες ουσιαστικά τελούνται χωρίς προφανή λόγο και ανεξάρτητα από αθλητικούς αγώνες, με τη βία να εξαπλώνεται επικίνδυνα ολοένα και περισσότερο στις χώρες τις Ευρώπης, όπως στην Ιταλία, Ισπανία, Τουρκία, Ολλανδία, Νορβηγία, Πολωνία, Κροατία, Ρωσία και Ελλάδα, ιδίως μετά το 1970.
\newline\setlength{\parindent}{20pt}\indent Ιστορικά Ο όρος χουλιγκανισμός
(``hooliganism``) άρχισε να χρησιμοποιείται απο τη δεκαετία του 1890, για να
χαρακτηρίσει τη συμπεριφορά συμμοριών των δρόμων στο Λονδίνο και έλκει την ονομασία του απο τον Ιρλανδό Πάτρικ Χούλιγκαν (Patrick Hooligan
), ο οποίος υπήρξε διαβόητος βάνδαλος και εγκληματίας στο Λονδίνο το 
1898.\newline 
Εκτός από την Ευρώπη, φαινόμενα χουλιγκανισμού έχουν παρατηρηθεί και στην
Αμερική, με τον όρο  να γίνεται διαχρονικός, εκφράζοντας και στις μέρες μας τον
``τρόπο ζωής`` κάποιων ανεγκέφαλων κλειστών, αλλά πολυπληθών ομάδων ανθρώπων, οι
οποίοι δρούν βίαια - σύμφωνα με την ψυχολογία της μάζας - κατά τον Φρόυντ, η
οποία (μάζα) λειτουργεί σύμφωνα με τον επηρεασμό και τη χειραγώγησή της, όπως υποστηρίζει ο γάλλος μαίτρ της κοινωνικής ψυχολογίας Γκουστάβ Λε Μπόν.
\subsection*{Αίτια και μορφές του μαθητικού χουλιγκανισμού}
\setlength{\parindent}{20pt}\indent Οι εκπαιδευτικές κοινότητες
Οι εκπαιδευτικές κοινότητες, ως μικρόκοσμος της κοινωνίας, δυστυχώς παρουσιάζουν
όλα τα χαρακτηριστικά και τις παθογένειες της οικογενειακής και κοινωνικής
δομής, από την οποία προέρχονται. Η απότομη μετάβαση από την αυταρχική στην
αντιαυταρχική εκπαίδευση των τελευταίων τριανταπέντε ετών, χωρίς να έχει περάσει
ένα στάδιο προετοιμασίας, έκανε το μαθητή από  τρομοκρατούμενο, τρομοκράτη. Η
επικράτηση ανεπαρκών εγχειριδίων μάθησης και η διδακτική ``ασυδοσία`` από
ανεκπαίδευτους ουσιαστικά ``δασκάλους`` και λανθασμένα εκπαιδευτικά συστήματα,
που επεβλήθησαν από το Υπουργείο Παιδείας των εκάστοτε Κυβερνήσεων, έκαναν τους
μαθητές όλων των βαθμίδων εκπαίδευσης ``πειραματόζωα``. Γι' αυτό και τα
κρούσματα πνευματικού χουλιγκανισμού, δηλαδή διανοητικής αλητείας, έχουν
επικίνδυνα πολλαπλασιασθεί στα σχολεία της Μέσης ιδιαίτερα Εκπαίδευσης, αφού το
μοντέλο της Παγκοσμοιοποίησης, θέλει παιδιά ρομπότ - εξαρτήματα και υπηρέτες του
συστήματος και όχι υγιώς σκεπτόμενους ανθρώπους.
\newline\setlength{\parindent}{20pt}\indent Το σημερινό επαχθές σύστημα
παιδείας στη χώρα μας δεν προωθεί τη γνώση, αλλά την επιβάλλει! Τα διδακτικά βιβλία
προσφέρουν έτοιμες αλήθειες και νεκρώνουν το γόνιμο στοχασμό. Η βαθμολογία του
μαθητή προέρχεται από την απομνημονευτική και όχι από την κριτική ικανότητά του,
γι' αυτό και τα βιβλία του τα αντιμετωπίζει εχθρικά και τα καταστρέφει με μανία.
Καλείται να αποκτήσει γνώση ξερής αποστήθισης, η οποία ουσιαστικά υπονομεύει την
παιδαγωγική πράξη και κάνει τους εκπαιδευτικούς χώρους στρατόπεδα πνευματικού
βασανισμού, όπου η μάθηση μετατρέπεται σε εκμάθηση, με αποκλειστικό σκοπό την
βαθμοθηρία\ldots\vspace{6 mm} Ο μαθητής δεν μαθαίνει πώς να σκέπτεται, αλλά με
τί να σκέπτεται. Δεν ασκεί την κρίση του για να βγάλει μόνος τα συμπεράσματά
του, αλλά αναγκάζεται να υιοθετήσει ιδέες, είτε αυτές τον εκφράζουν, είτε
όχι.
\newline\setlength{\parindent}{20pt}\indent Έλεγε ο Αριστοτέλης: ``Οι νέοι φιλώσι τε άγαν και μισούσιν άγαν και τ' άλλα πάντα
ομοίως`` (Οι νέοι είναι υπερβολικοί σε όλες τις εκδηλώσεις τους). Η πνευματική
αυτή ``δικτατορία`` λοιπόν του εκπαιδευτικού συστήματος μετατρέπει τα παιδιά σε
σκυλιά ράτσας ``ντόμπερμαν``, τα οποία, όταν διογκωθεί ο εγκέφαλός τους,
στρέφονται και κατά του εκγυμναστή τους. Το ίδιο και τα παιδιά,  με την
ανωριμότητα και την επιπολαιότητα που τα διακρίνουν, πιστεύουν ότι, με την
αυθάδεια και τη χειροδικία εναντίον των διδασκόντων και με την απερίσκεπτη
καταστροφή βιβλίων, εκπαιδευτικού υλικού και κτηριακών εγκαταστάσεων των
σχολείων τους, αντιδρούν και εκδικούνται όσους και ό,τι τους μαθαίνει κάτι,
χωρίς όμως - δυστυχώς - να τους διεγείρει το στοχασμό\ldots\vspace{6 mm} Είναι
λοιπόν ορατός ο κίνδυνος αναβίωσης του πνεύματος του γκαιμπελισμού, αφού οι μαθητές, που
καταστρέφουν τα πάντα στο σχολείο τους, μπορούν, με τον ίδιο τρόπο, να
μετατραπούν σε διώκτες και άλλων πραγμάτων,  με τα οποία διαφωνούν ή που τα
θεωρούν περιττά και ανούσια. Δυστυχώς, ο νεοφασισμός, με την μορφή του
αχαλίνωτου κομματισμού, έχει αρχίσει να χτυπά τις πόρτες των σχολείων
μας\ldots\vspace{6 mm}
\newline\setlength{\parindent}{20pt}\indent Ο χουλιγκανισμός μέσα στα σχολεία
οφείλεται - πέραν των όσων ανεφέρθησαν - και σε πολλά άλλα αίτια και εκφράζεται
με διάφορες μορφές, που θα αναπτύξουμε συνοπτικά παρακάτω: Γενικά τα αγόρια
είναι πιο επιθετικά από τα κορίτσια, με πιο σκληρή συμπεριφορά, που αρχίζει από
την προσχολική ηλικία και φτάνει στο ζενίθ της στην εφηβική ηλικία των 15 έως 18
ετών. Η επιθετικότητά τους μπορεί να είναι \textbf{άμεση και συντελεστική}
(δηλαδή ο μαθητής απειλεί τους άλλους) ή \textbf{άμεση και αντιδραστική} (συχνά
απειλεί τους άλλους για να πετύχει το στόχο του ή απειλεί όσους τον απειλούν).
Υπάρχει όμως και η \textbf{έμμεση και συντελεστική} επιθετικότητα (όπου ο
μαθητής κουτσομπολεύει και διαδίδει φήμες για τους άλλους) και η \textbf{έμμεση
και αντιδραστική} επιθετικότητα (όπου, όταν γίνεται έξαλλος με τους άλλους,
συχνά κουτσομπολεύει ή διαδίδει φήμες γι' αυτούς).
\newline\setlength{\parindent}{20pt}\indent Πολλές φορές, τα παιδιά
εκπαιδεύονται από την οικογένειά στους να αντιδρούν επιθετικά, προκειμένου να
κυριαρχήσουν στην μαθητική ομάδα της τάξης τους, ή εξωθούνται στη βία ως
αντίδραση στη σκληρή και άδικη μεταχείριση από τους γονείς τους, οι οποίοι είτε
αδιαφορούν γι' αυτά, είτε τα κακοποιούν λεκτικά και σωματικά, είτε κόβουν τις
γέφυρες αγάπης και επικοινωνίας με αυτά, μετά από ένα διαζύγιο.
\newline\setlength{\parindent}{20pt}\indent Η πολύωρη απουσία των γονέων από το
σπίτι για δουλειά, η σωματική και ψυχική κόπωσή τους και οι διακρίσεις ανάμεσα
στα παιδιά τους δημιουργούν σ' αυτά αισθήματα θυμού και απογοήτευσης, ενώ η
πίεση για ν' ανταποκριθούν τα παιδιά σε πολλαπλά καθήκοντα, χωρίς να
απολαμβάνουν ελεύθερο χρόνο για παιχνίδι και δημιουργική απασχόληση, δημιουργούν
εκρηκτικές αντιδραστικές συμπεριφορές από μέρος τους, που εκτονώνονται με βία,
πάνω στα άλλα παιδιά ή και στο διδακτικό προσωπικό και στις κτηριακές υποδομές
του σχολείου.
\end{document}