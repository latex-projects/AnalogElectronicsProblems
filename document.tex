\documentclass[12pt,a4paper]{book}
\usepackage{fontspec,titlesec,fancyhdr}
\usepackage{showframe}
\usepackage{layout}
% \usepackage{}
\pagestyle{fancy}
\lhead{}
\chead{}
\rhead{}
\lfoot{}
\cfoot{}
% \fancyhf{}
\rfoot{\thepage}
\renewcommand\headrule{}
\usepackage[margin=0.7in,footskip=0.25in]{geometry}
\textwidth = 510pt %right page margin
\setmainfont{Arial}
% \makeatletter
% \newcommand\thefontsize[1]{{#1 The current font size is: \f@size pt\par}}
% \makeatother
\begin{document}
\layout{}
%\section{Font Sizes}
% \sffamily
% \thefontsize\tiny
% \thefontsize\scriptsize
% \thefontsize\footnotesize
% \thefontsize\small
% \thefontsize\normalsize
% \thefontsize\large
% \thefontsize\Large
% \thefontsize\LARGE
% \thefontsize\huge
% \thefontsize\Huge
\section*{Ο χουλιγκανισμός στα σχολεία}
\subsection*{Ορισμός και ετυμολογία} \parindent 0pt Ως χουλιγκανισμό αρχικά
αναφέρεται η ανάρμοστη, βίαιη και αντικοινωνική συμπεριφορά οπαδών αθλητικών ομάδων, που οδηγεί στη διατάραξή της τάξης και
συνδέεται με βαθύτερα κοινωνικά αίτια, έχοντας αντίκτυπο σε όλες τις δομές της
κοινωνικοπολιτικής ζωής της χώρας. Η ρατσιστική αυτή διάθεση και ομαδικές
συγκρούσεις οπαδών εντός και εκτός γηπέδων καταλήγουν πολλές φορές σε δολοφονίες
, οι οποίες ουσιαστικά τελούνται χωρίς προφανή λόγο και ανεξάρτητα από
αθλητικούς αγώνες, με τη βία να εξαπλώνεται επικίνδυνα ολοένα και περισσότερο
στις χώρες τις Ευρώπης, όπως στην Ιταλία, Ισπανία, Τουρκία, Ολλανδία, Νορβηγία,
Πολωνία, Κροατία, Ρωσία και Ελλάδα, ιδίως μετά το 1970.\paragraph
Ο όρος χουλιγκανισμός (hooliganism), άρχισε να χρησιμοποιείται απο τη δεκαετία 
του 1890, για να χαρακτηρίσει τη συμπεριφορά συμμοριών των δρόμων , στο Λονδίνο
και έλκει την ονομασία του απο τον Ιρλανδό Πάτρικ Χούλιγκαν (Patrick Hooligan 
- Hoolihan), ο οποίος υπήρξε διαβόητος βάνδαλος και εγκληματίας στο Λονδίνο το 
1898.\paragraph
εΕκτός από την Ευρώπη, φαινόμενα χουλιγκανισμού έχουν παρατηρηθεί και στην
Αμερική, με τον όρο  να γίνεται διαχρονικός.
\end{document}