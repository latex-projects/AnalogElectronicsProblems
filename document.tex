\documentclass[12pt,a4paper,oneside]{book}
\usepackage{fontspec,fancyhdr,
ragged2e,blindtext,indentfirst,graphicx,
xunicode,xltxtra,xgreek,lipsum,siunitx,
hyphenat,etoolbox,tocloft}
%\usepackage[nodayofweek]{datetime}
\usepackage{datetime}
%\newdate{date}{}{11}{2014}
%\date{\displaydate{date}}
\usepackage[nottoc]{tocbibind}
\usepackage[shortcuts]{extdash} % hyphernate using dash
\usepackage[center]{titlesec}
\usepackage[english,greek]{babel}
%\usepackage{showframe}
\setlength{\parindent}{2em}
\renewcommand{\cftsecfont}{\scshape} % table of contents subsection smaller size
%\usepackage{showframe}
\newdateformat{Mydate}{\monthname[\THEMONTH] \THEYEAR}
\renewcommand{\baselinestretch}{1.5} % line spacing in a paragraph
\renewcommand{\cftchapleader}{\cftdotfill{\cftdotsep}} % for dots in chapters 
% table of contents
%\raggedbottom
% xgreek package required for greek hypernation
%\usepackage{showframe}
%\usepackage{layout}

\hyphenation{Cyber-Bully-ing}

\hyphenation{Cy-ber-spa-ce}

\hyphenation{Nati-on-al}

\hyphenation{Flam-ing}

% \usepackage{}
\pagestyle{fancy}
\cfoot{}




% \fancyhf{}
%\thepage
%\rfoot{\rightskip=-0.8in \thepage}
%\renewcommand{\headrulewidth}{0pt}

%\rfoot{\thepage}
%\raggedbottom
\renewcommand{\headrulewidth}{0pt}
\fancypagestyle{plain}{  % plain for pages that have chapters
\renewcommand{\headrulewidth}{0pt}
\fancyhf{}
\fancyfoot{}

%\cleardoublepage
%\setcounter{page}{2}
% \rfoot{\thepage}
% \fancyfoot[RO]{\leftskip=-0.27in \thepage }
% \fancyfoot[RE]{\rightskip=5.32in \thepage}
%\fancyfoot[RE]{\leftskip=5.22in \thepage} 
%\rfoot{\leftskip=6.32in\thepage}
\fancyfoot{\hspace{-1.1pt}\setlength{\footskip}{20pt}\leftskip=6.32in\fancyplain{}\thepage}
}

\fancyfoot{\hspace{-1pt}\setlength{\footskip}{20pt}\leftskip=6.32in\fancyplain{}\thepage}
%\fancyfoot[RE]{\leftskip=6.32in \thepage}
%     \makeatletter
%     \renewcommand*{\pagenumbering}[2]{%
%       \gdef\thepage{\csname @#1\endcsname\c@page}%
%     }
%     \makeatother

%\renewcommand\headrule{}          %35
%\footskip = 665pt
%\textheight = 643pt
%\textwidth = 610pt
%\headsep = 09pt

%\setlength{\parskip}{15pt plus 1pt minus 2pt}
%\setlength{\parskip}{-20.5pt}%
%\setlength{\parindent}{50pt}%

%\textwidth = 410pt %right page margin
%\setmainlanguage[variant=mono]{greek}
% \newcommand{\en}[1]{\setlanguage{american}#1\setlanguage{moderngreek}}
\setmainfont[Ligatures=TeX]{Times New Roman}
% \makeatletter
% \newcommand\thefontsize[1]{{#1 The current font size is: \f@size pt\par}}
% \makeatother


%\titlespacing\section{0pt}{1 cm}{2pt}
\usepackage[footskip=0.35in,left=20mm,right=20mm,top=25mm,bottom=35mm]{geometry}
\setlength{\parindent}{40pt} % spacing between paragraphs
\titlespacing\chapter{6pt}{6 pt plus 3pt minus 5pt}{6 pt plus 6pt minus 3pt}

\titlespacing\section{6pt}{3 pt plus 6pt minus 3pt}{6 pt plus 3pt minus 3pt}
\titlespacing\subsection{6pt}{3 pt plus 6pt minus 3pt}{6 pt plus 3pt minus
3pt}
\sloppy
\cleardoublepage
%\setcounter{page}{3}
%\title{Sections and Chapters}
%\makeindex
\setcounter{secnumdepth}{0}
\begin{document}
\begin{titlepage}

\begin{center}
\Large ΤΕΙ ΚΕΝΤΡΙΚΗΣ ΜΑΚΕΔΟΝΙΑΣ \\
\large ΣΧΟΛΗ ΤΕΧΝΟΛΟΓΙΚΩΝ ΕΦΑΡΜΟΓΩΝ\\
ΤΜΗΜΑ ΜΗΧΑΝΙΚΩΝ ΠΛΗΡΟΦΟΡΙΚΗΣ \\

\vspace*{5cm}
\Huge \,\,\,Ο χουλιγκανισμός στα σχολεία

\large Eργασία στο μάθημα των Παιδαγωγικών

 του 7ου εξαμήνου
 \vspace*{4cm}
 
NameNameNameNameN - 0000

NameNameNameNameN - 0000

NameNameNameNameN - 0000

NameNameNameNameN - 0000 

 \vspace*{2.5cm}
 \nopagebreak
Σέρρες\\
Νοέμβριος 2014

\end{center}

\end{titlepage}
%\newgeometry{footskip=0.35in,left=20mm,right=20mm,top=20mm,bottom=35mm}
%\maketitle

%\maketitle
%  interwordspace: \the\fontdimen2\font \
%  interwordstretch: \the\fontdimen3\font \
%  emergencystretch: \the\emergencystretch\par
 %\blindtext
%\layout{}
%\section{Font Sizes}
% \sffamily
% \thefontsize\tiny
% \thefontsize\scriptsize
% \thefontsize\footnotesize
% \thefontsize\small
% \thefontsize\normalsize
% \thefontsize\large
% \thefontsize\Large
% \thefontsize\LARGE
% \thefontsize\huge
% \thefontsize\Huge

\chapter*{Ο χουλιγκανισμός στα σχολεία}
\addcontentsline{toc}{chapter}{Ο χουλιγκανισμός στα σχολεία}
\addcontentsline{toc}{section}{Ορισμός και ετυμολογία του όρου}
\section*{Ορισμός και ετυμολογία του όρου}
\indent Ως  χουλιγκανισμός, αρχικά, αναφέρεται η ανάρμοστη,
βίαιη και αντικοινωνική συμπεριφορά οπαδών, αθλητικών ομάδων, που οδηγεί στη διατάραξη της τάξης και συνδέεται με βαθύτερα κοινωνικά
αίτια, έχοντας αντίκτυπο σε όλες τις δομές της κοινωνικοπολιτικής ζωής της
χώρας. Η ρατσιστική αυτή διάθεση και οι ομαδικές συγκρούσεις οπαδών εντός και 
εκτός γηπέδων, καταλήγουν, πολλές φορές, σε δολοφονίες, οι οποίες ουσιαστικά
τελούνται χωρίς προφανή λόγο και ανεξάρτητα από αθλητικούς αγώνες, με τη βία να εξαπλώνεται επικίνδυνα ολοένα και περισσότερο στις χώρες τις Ευρώπης, όπως στην Ιταλία, Ισπανία, Τουρκία, Ολλανδία, Νορβηγία, Πολωνία, Κροατία, Ρωσία και Ελλάδα, ιδίως μετά το 1970.

\indent Ιστορικά o όρος χουλιγκανισμός
(``hooliganism``) άρχισε να χρησιμοποιείται από τη δεκαετία του \num{1890}, για να χαρακτηρίσει τη συμπεριφορά συμμοριών των δρόμων στο Λονδίνο και έλκει την ονομασία του από τον Ιρλανδό Πάτρικ Χούλιγκαν (Patrick Hooligan), ο οποίος υπήρξε διαβόητος βάνδαλος και εγκληματίας στο Λονδίνο το \num{1898}.

\indent Εκτός από την Ευρώπη, φαινόμενα χουλιγκανισμού έχουν παρατηρηθεί και
στην Αμερική, με τον όρο  να γίνεται διαχρονικός, εκφράζοντας και στις μέρες μας τον ``τρόπο ζωής`` κάποιων ανεγκέφαλων κλειστών, αλλά πολυπληθών ομάδων ανθρώπων, οι οποίοι δρούν βίαια - σύμφωνα με την ψυχολογία της μάζας - κατά τον Φρόυντ, η οποία (μάζα) λειτουργεί σύμφωνα με τον επηρεασμό και τη χειραγώγησή της, όπως υποστηρίζει ο Γάλλος μαίτρ της κοινωνικής ψυχολογίας Γκουστάβ Λε Μπόν.

\section*{Αίτια και μορφές του μαθητικού χουλιγκανισμού}
\addcontentsline{toc}{section}{Αίτια και μορφές του μαθητικού χουλιγκανισμού}
\indent Οι εκπαιδευτικές κοινότητες, ως μικρόκοσμος της κοινωνίας, δυστυχώς
παρουσιάζουν όλα τα χαρακτηριστικά και τις παθογένειες της οικογενειακής και κοινωνικής δομής, από την οποία προέρχονται. Η απότομη μετάβαση από την αυταρχική στην
αντιαυταρχική εκπαίδευση των τελευταίων τριανταπέντε ετών, χωρίς να έχει περάσει
ένα στάδιο προετοιμασίας, έκανε το μαθητή από  τρομοκρατούμενο, τρομοκράτη. Η
επικράτηση ανεπαρκών εγχειριδίων μάθησης και η διδακτική ``ασυδοσία`` από
ανεκπαίδευτους ουσιαστικά ``δασκάλους`` και λανθασμένα εκπαιδευτικά συστήματα,
που επεβλήθησαν από το Υπουργείο Παιδείας των εκάστοτε Κυβερνήσεων, έκαναν τους
μαθητές όλων των βαθμίδων εκπαίδευσης ``πειραματόζωα``. Γι' αυτό και τα
κρούσματα πνευματικού χουλιγκανισμού, δηλαδή διανοητικής αλητείας, έχουν
επικίνδυνα πολλαπλασιασθεί στα σχολεία της Μέσης ιδιαίτερα Εκπαίδευσης, αφού το
μοντέλο της Παγκοσμιοποίησης, θέλει παιδιά ρομπότ - εξαρτήματα και υπηρέτες του
συστήματος και όχι υγιώς σκεπτόμενους ανθρώπους\ldots

\indent Το σημερινό επαχθές σύστημα Παιδείας στη Χώρα μας δεν προωθεί τη γνώση,
αλλά την επιβάλλει! Τα διδακτικά βιβλία προσφέρουν έτοιμες αλήθειες και
νεκρώνουν το γόνιμο στοχασμό. Η βαθμολογία του μαθητή προέρχεται από την
απομνημονευτική και όχι από την κριτική ικανότητά του, γι' αυτό και τα βιβλία
του τα αντιμετωπίζει εχθρικά και τα καταστρέφει με μανία. Ο μαθητής καλείται να
αποκτήσει γνώση ξερής αποστήθισης, η οποία ουσιαστικά υπονομεύει την παιδαγωγική
πράξη και κάνει τους εκπαιδευτικούς χώρους στρατόπεδα πνευματικού βασανισμού,
όπου η μάθηση μετατρέπεται σε εκμάθηση, με αποκλειστικό σκοπό την
βαθμοθηρία\ldots \,\,\, Ο μαθητής δεν μαθαίνει πώς να σκέπτεται, αλλά με τί
να σκέπτεται.\raggedbottom Δεν ασκεί την κρίση του για να βγάλει μόνος τα
συμπεράσματά του, αλλά αναγκάζεται να υιοθετήσει ιδέες, είτε αυτές τον εκφράζουν, είτε όχι.

\indent Έλεγε ο Αριστοτέλης: ``Οι νέοι φιλώσι τε άγαν και μισούσιν άγαν και τ'
άλλα πάντα ομοίως`` (Οι νέοι είναι υπερβολικοί σε όλες τις εκδηλώσεις τους). Η
πνευματική αυτή ``δικτατορία`` λοιπόν του εκπαιδευτικού συστήματος μετατρέπει τα
παιδιά σε σκυλιά ράτσας ``ντόμπερμαν``, τα οποία, όταν διογκωθεί ο εγκέφαλός
τους, στρέφονται και κατά του εκγυμναστή τους. Το ίδιο και τα παιδιά, με την
ανωριμότητα και την επιπολαιότητα που τα διακρίνουν, πιστεύουν ότι, με την
αυθάδεια και τη χειροδικία εναντίον των διδασκόντων και με την απερίσκεπτη
καταστροφή βιβλίων, εκπαιδευτικού υλικού και κτηριακών εγκαταστάσεων των
σχολείων τους, αντιδρούν και εκδικούνται όσους και ό,τι τους μαθαίνει κάτι,
χωρίς όμως - δυστυχώς - να τους διεγείρει το στοχασμό\ldots\,\,\, Είναι
λοιπόν ορατός ο κίνδυνος αναβίωσης του πνεύματος του γκαιμπελισμού, αφού οι
μαθητές, που καταστρέφουν τα πάντα στο σχολείο τους, μπορούν, με τον ίδιο τρόπο,
να μετατραπούν σε διώκτες και άλλων πραγμάτων,  με τα οποία διαφωνούν ή που τα
θεωρούν περιττά και ανούσια. Δυστυχώς, ο νεοφασισμός, με την μορφή του
αχαλίνωτου κομματισμού, έχει αρχίσει να χτυπά τις πόρτες των σχολείων μας\ldots

\indent Ο χουλιγκανισμός μέσα στα σχολεία οφείλεται - πέραν των όσων ανεφέρθησαν
- και σε πολλά άλλα αίτια και εκφράζεται με διάφορες μορφές, που θα αναπτύξουμε
συνοπτικά παρακάτω. Γενικά τα αγόρια είναι πιο επιθετικά από τα κορίτσια, με πιο
σκληρή συμπεριφορά, που αρχίζει από την προσχολική ηλικία και φθάνει στο ζενίθ
της στην εφηβική ηλικία των \num{15} έως \num{18} ετών. Η επιθετικότητά τους
μπορεί να είναι \textbf{άμεση και συντελεστική} (δηλαδή ο μαθητής απειλεί τους
άλλους) ή \textbf{άμεση και αντιδραστική} (συχνά απειλεί τους άλλους για να
πετύχει το στόχο του ή απειλεί όσους τον απειλούν). Υπάρχει όμως και η
\textbf{έμμεση και συντελεστική} επιθετικότητα (όπου ο μαθητής κουτσομπολεύει
και διαδίδει φήμες για τους άλλους) και η \textbf{έμμεση και αντιδραστική}
επιθετικότητα (όπου, όταν γίνεται έξαλλος με τους άλλους, συχνά κουτσομπολεύει ή
διαδίδει φήμες γι' αυτούς).

\indent Πολλές φορές, τα παιδιά εκπαιδεύονται από την οικογένειά τους να
αντιδρούν επιθετικά, προκειμένου να κυριαρχήσουν στη μαθητική ομάδα της τάξης
τους, ή εξωθούνται στη βία, ως αντίδραση στη σκληρή και άδικη μεταχείριση από
τους γονείς τους, οι οποίοι είτε αδιαφορούν γι' αυτά, είτε τα κακοποιούν λεκτικά
και σωματικά, είτε κόβουν τις γέφυρες αγάπης και επικοινωνίας με αυτά, μετά από
ένα διαζύγιο.

\indent Η πολύωρη απουσία των γονέων από το σπίτι για δουλειά, η σωματική και
ψυχική κόπωσή τους και οι διακρίσεις ανάμεσα στα παιδιά τους δημιουργούν σ' αυτά
αισθήματα θυμού και απογοήτευσης, ενώ η πίεση για ν' ανταποκριθούν τα παιδιά σε
πολλαπλά καθήκοντα, χωρίς να απολαμβάνουν ελεύθερο χρόνο για παιχνίδι και
δημιουργική απασχόληση, δημιουργούν εκρηκτικές αντιδραστικές συμπεριφορές από
μέρος τους, που εκτονώνονται με βία, πάνω στα άλλα παιδιά ή και στο διδακτικό
προσωπικό και στις κτηριακές υποδομές του σχολείου.

\indent Ο μιμητισμός των παιδιών αναπαράγει τη βία, που βιώνουν στην οικογένεια,
ενώ συχνά τα Μ.Μ.Ε. παίζουν καταλυτικό ρόλο και εξωθούν τα παιδιά στη βία, με τα αρνητικά πρότυπα που προβάλλουν διαρκώς
μεγαλοποιημένα\ldots

\indent Πέρα όμως από το οικογενειακό περιβάλλον, οι νοσηρές συνθήκες διαβίωσης
στις μεγαλουπόλεις, με την αποκοπή των παιδιών από την ηρεμία της φύσης, η απομάκρυνσή τους από φυτά και ζώα, η κακή
και βλαβερή διατροφή τους με junk food, το στήσιμο και η ακινησία τους μπροστά
στην τηλεόραση ή τον υπολογιστή, η έλλειψη επαφής με φίλους, οι ηθικοί κίνδυνοι
που καραδοκούν, ο ωχαδερφισμός, ο ρατσισμός, ο φανατισμός, η έλλειψη
αξιοκρατίας, το άγχος των μαθημάτων και η πιεστική καθημερινότητα, επιτείνουν τα
αισθήματα περιθωριοποίησης, ανασφάλειας, μοναξιάς και μελαγχολίας των παιδιών,
των εφήβων και των νέων, με αποτέλεσμα αυτοί να γίνονται καχύποπτοι, εγωιστές,
κτητικοί και αυτοκαταστροφικοί.

\indent Η σημερινή οικονομική κρίση, πέρα από τη φτώχεια, την ανέχεια και τη
δυστυχία, έχει δημιουργήσει και πολλαπλά άλλα προβλήματα, αφού κυρίως είναι κρίση αξιών και σκληρού και απάνθρωπου
ανταγωνισμού. Αντί της ευγενούς άμιλλας, κυριαρχεί το δίκαιο του ισχυρού, αντί
της πειθαρχίας, κυριαρχεί η ασυδοσία, αντί του σεβασμού, της συνεργασίας, της
συλλογικότητας, της εντιμότητας και του ήθους, κυριαρχούν αισθήματα αποξένωσης,
εχθρότητας, αντιπαλότητας και καταπάτησης κάθε νομίμου δικαιώματος των άλλων. Η
οικονομική δυσπραγία αύξησε τους φόνους, τις κλοπές, τις αυτοκτονίες, τα
αδιέξοδα, την επιθετικότητα, τη βία και κάθε είδους παραβατική συμπεριφορά, αλλά
και το φόβο, το θυμό και την ανασφάλεια.

\indent Ο κοινωνικός ιστός έχει προ πολλού διαρραγεί. Οι κοινωνικές πληγές είναι
αμέτρητες: ανεπαρκής πρόνοια  για μεγάλους και μικρούς, για νέους και γέροντες, για αναπήρους και πάσχοντες. Η ανεργία και η φτωχοποίηση του λαού οδήγησαν - κυρίως τους εφήβους και τους
νέους - σε απώλεια της αξιοπρεπούς διαβίωσής τους, σε πενία και στέρηση, σε
ματαίωση των σχεδίων τους για το μέλλον. Ως αποτέλεσμα ήρθε η μαζοποίησή τους, η
επιθετικότητα και η οχλοκρατική συμπεριφορά. Οι ηθικές αξίες έχουν καταρρεύσει,
τα πρότυπα εξαφανίσθηκαν και η χαλάρωση των ηθών οδηγεί αναπόφευκτα σε κάθε
μορφής παραβατικότητα.

\indent Το χρήμα σήμερα έχει θεοποιηθεί.
Καταναλωτικό, ευδαιμονιστικό και υλιστικό πνεύμα κυριαρχεί παντού. Όλα
διέπονται από το συμφέρον. Προς τί λοιπόν η απορία για τον κατήφορο των παιδιών
και των εφήβων; Το μήλο πέφτει κάτω από τη μηλιά\ldots

\indent Μια κατηγορία παιδιών με
επιθετική συμπεριφορά είναι και αυτά τα παιδιά, που τα αποδοκιμάζει και τα απορρίπτει η
σχολική κοινότητα. Έτσι λοιπόν και αυτά αποξενώνονται από τους συνομηλίκους
τους, αποκτούν χαμηλή αυτοεκτίμηση, στερούνται επαρκών κοινωνικών δεξιοτήτων και
είναι μή συνεργάσιμα. Πολλές φορές, τα παιδιά αυτά, από θύματα, γίνονται θύτες.
Με τη βίαιη και αντικοινωνική συμπεριφορά τους, ιδιαίτερα όταν μεγαλώσουν,
γίνονται επιθετικά, εκδικητικά και παραβατικά και καταλήγουν, ως χρήστες
ναρκωτικών ή αλκοολικοί, με διαταραχές του ψυχισμού τους και τάση να
εντάσσονται σε ομάδες άλλων χούλιγκανς.

\indent Έρευνες επιστημόνων έχουν καταδείξει ότι, η σεξουαλική κακοποίηση, η
ενδοοικογενειακή βία και γενικότερα η έκθεση των παιδιών σε βίαια περιβάλλοντα, όπου υπάρχει έλλειψη στοργής και φροντίδας,
προάγουν την επιθετικότητα και τα προβλήματα συμπεριφοράς των παιδιών στο
σχολείο.

\indent Σήμερα, οι επιστήμονες έχουν
εντοπίσει, επίσης, ένα πλήθος οργανικών και βιολογικών παραγόντων, οι οποίοι
θεωρούνται ως προδιαθεσικοί παράγοντες, για την εμφάνιση περιστατικών επιθετικής
ή βίαιης συμπεριφοράς στα παιδιά και στους εφήβους. Στους παράγοντες αυτούς
συγκαταλέγονται: \textbf{σωματικές παθήσεις} (όπως: εγκεφαλοπάθειες, όγκοι του
εγκεφάλου, κρανιοεγκεφαλικές κακώσεις), \textbf{διαταραχές της ρύθμισης των
νευροδιαβιβαστών} (όπως: έλλειψη ή ελάττωση της σεροτονίνης), \textbf{διαταραχές
της νευροενδοκρινικής ρύθμισης} (όπως: αύξηση της ορμόνης τεστοστερόνης), καθώς
επίσης \textbf{ανεπάρκεια μετάλλων και ιχνοστοιχείων} (είτε λόγω μη
ισορροπημένης διατροφής, είτε λόγω κακής απορρόφησης). Οι παράγοντες αυτοί είναι δυνατόν να επηρεάσουν την
ικανότητα μάθησης και επικοινωνίας του ατόμου με το κοινωνικό του περιβάλλον.

\indent Επίσης, \textbf{γνωστικοί και συναισθηματικοί παράγοντες} (όπως: νοητική
στέρηση, σοβαρές ψυχικές διαταραχές, διαταραχές προσοχής με υπερκινητικότητα,
ελλείμματα βασικών κοινωνικών δεξιοτήτων, χαμηλή αυτοεκτίμηση, ευέξαπτη
ιδιοσυγκρασία ή επιθετική ροπή, διάψευση προσδοκιών και ματαίωση στόχων ή έντονα
δυσάρεστες καταστάσεις πόνου και δυσφορίας, που δημιουργούνται σε κάποιον από
ακατάλληλες περιβαλλοντικές συνθήκες, καθώς και η χρήση ψυχοτρόπων ουσιών, όπως
είναι το αλκοόλ, τα ναρκωτικά, ακόμα και η καφεΐνη) αυξάνουν την επιθετικότητα,
κυρίως διά μέσου της ενίσχυσης του αρνητικού θυμικού.
\addcontentsline{toc}{section}{Το φαινόμενο του bullying}
\section*{Το φαινόμενο του bullying}
\indent Μία μορφή επιθετικότητας και ενδοσχολικής βίας μεταξύ των μαθητών -
κυρίως του Δημοτικού και του Γυμνασίου και λιγότερο του Λυκείου - είναι και ο εκφοβισμός (κοινώς bullying). Είναι ένα
διεθνές φαινόμενο επαναλαμβανόμενης, απρόκλητης, σκόπιμης και συνειδητής
πρόθεσης πρόκλησης βλάβης ή φόβου σε κάποιον, που έχει πάρει στις μέρες μας
ανησυχητικές διαστάσεις, με σοβαρές επιπτώσεις για ένα μεγάλο αριθμό μαθητών της
Πρωτοβάθμιας και της Δευτεροβάθμιας Εκπαίδευσης, προκαλώντας ιδιαίτερο
προβληματισμό στους γονείς, στους δασκάλους και στους παιδαγωγούς ερευνητές.

\indent Πρόσφατη έρευνα, η οποία
διενεργήθηκε από τον Παγκόσμιο Οργανισμό Υγείας σε τριάντα Χώρες, έδειξε ότι, το
ποσοστό των μαθητών ηλικίας \num{11} έως \num{15} ετών, οι οποίοι επιδίδονται σε
συμπεριφορές εκφοβισμού συνομηλίκων τους, κυμαίνεται από \num{9} \% έως
\num{73} \%, ενώ άλλη έρευνα σε \num{60} Xώρες αποκάλυψε ότι, το \num{37,4} \%
των παιδιών έχει υποστεί εκφοβισμό, τουλάχιστον μία φορά το μήνα!

\indent Εξαιτίας της επαναληπτικής φύσης του bullying και της αδυναμίας του
θύματος να υπερασπισθεί τον εαυτό του, ο εκφοβισμός αποτελεί μια εξαιρετικά επιβλαβή μορφή θυματοποίησης (Victimization)
και μία πράξη δειλίας, η οποία αρχίζει συνήθως με τον εκφοβισμό στην πρωτοβάθμια
εκπαίδευση και στις πρώτες τάξεις της δευτεροβάθμιας εκπαίδευσης και αργότερα
εξελίσσεται σε παρενόχληση και βία στις διαπροσωπικές συναισθηματικές σχέσεις,
που ακολουθούνται κατά την ενήλικη ζωή από ένα πλήθος ακαδημαϊκών, κοινωνικών
και ψυχοσωματικών συνεπειών (άγχος, κατάθλιψη, ανασφάλεια, σχολική φοβία,
χαμηλή αυτοεκτίμηση, αυτοκτονικές τάσεις, απομόνωση, πονοκέφαλος, έμετος,
διαταραχές ύπνου, χαμηλές σχολικές επιδόσεις, δυσκολία στη σχολική και
κοινωνική προσαρμογή και στις σχέσεις με τους συνομηλίκους, διαταραχή
μετατραυματικού στρές).

\indent Το σχολικό bullying μπορεί να περιλαμβάνει: \textbf{1)} τον
\textbf{άμεσο εκφοβισμό}, που είναι: \textbf{α)} \textbf{σωματικός} (δηλαδή: δάγκωμα, τράβηγμα μαλλιών, χτύπημα, κλωτσιά, κλείδωμα σε δωμάτιο,
τσίμπημα, γρονθοκόπημα, σπρώξιμο, γρατζούνισμα, φτύσιμο, άλλες μορφές σωματικής
επίθεσης, καταστροφή ή κλοπή προσωπικών αντικειμένων). \textbf{β)}
\textbf{Άμεσος λεκτικός εκφοβισμός} (δηλαδή: προσβλητικά σχόλια, ενοχλητικά
τηλεφωνήματα, απειλές, προσβλητικές προσφωνήσεις, φυλετικά σχόλια ή πειράγματα, σεξουαλικά
υπονοούμενα, σκληρά σχόλια ή βρισιές, διάδοση ψευδών και κακόβουλων φημών,
κλπ) και \textbf{γ)} \textbf{εκφοβισμός, που δεν είναι ούτε σωματικός ούτε
λεκτικός} (δηλαδή: προσβλητικές γκριμάτσες ή άσεμνες χειρονομίες). Επίσης
περιλαμβάνει: \textbf{2)} τον \textbf{έμμεσο εκφοβισμό} (δηλαδή: χειραγώγηση και
καταστροφή φιλικών σχέσεων, συστηματική περιθωριοποίηση, αγνόηση και απομόνωση, αποστολή ανωνύμων
συνήθως σημειωμάτων, κλπ) και \textbf{3)} \textbf{διάφορες εγκληματικές
συμπεριφορές} (όπως: επίθεση με όπλο, σοβαρή σωματική βλάβη, απειλή για πρόκληση βλάβης,
ληστεία, σεξουαλική επίθεση).

\indent Η μορφή αυτή του σχολικού εκφοβισμού αφορά σε ίσα περίπου ποσοστά
κοριτσιών, που εκφοβίζουν κορίτσια και αγοριών, που εκφοβίζουν αγόρια ή κορίτσια, με τη συνηθέστερη μορφή εκφοβισμού, που είναι η
άμεση λεκτική επιθετικότητα, προκειμένου, τα μεν αγόρια να κυριαρχήσουν στη
μεγάλη ομάδα των συνομηλίκων συμμαθητών τους, τα δε κορίτσια να διαφυλάξουν
τον περιορισμένο αριθμό των μελών της ομάδας τους,  αποκλείοντας κάποια άλλα
μέλη\ldots

\indent Πρωτού αναφερθούμε στους αιτιολογικούς παράγοντες του σχολικού
εκφοβισμού, είναι απαραίτητο να γράψουμε δύο λόγια και για τα χαρακτηριστικά των ``θεατών`` - συνομηλίκων των θυτών και
των θυμάτων του bullying στις σχολικές αίθουσες, καθώς και για το ρόλο, που
αυτοί διαδραματίζουν σ' αυτό.

\indent Πολλές έρευνες δείχνουν ότι, οι θεατές (κυρίως συμμαθητές και
συνομήλικοι) είναι παρόντες σε επεισόδια εκφοβισμού στο σχολικό χώρο, σε ένα ποσοστό περίπου \num{85} \%. Αυτά τα υψηλά ποσοστά φανερώνουν ότι, περιστατικά εκφοβισμού είναι πιο πιθανό να συμβούν, όταν τριγύρω βρίσκονται άλλοι συνομήλικοι  και αυτό συμβαίνει, επειδή σε επεισόδια εκφοβισμού, πέραν του
θύτη και του θύματος, ένα ποσοστό \num{30} \% των παρευρισκομένων συμμαθητών,
έχει τελικά ενεργό συμμετοχή στα επεισόδια.

\indent Αυτή η συμμετοχή, που μπορεί να έχει ακόμη και τη μορφή γέλιων ή
ζητωκραυγών, ενισχύει την συμπεριφορά των θυτών και την κάνει όχι μόνο να είναι αποδεκτή, αλλά και να υποστηρίζεται ενεργά από τους συνομηλίκους. Έτσι λοιπόν, οι παριστάμενοι σε περιστατικά εκφοβισμού διακρίνονται σ' αυτούς, που προσχωρούν και λαμβάνουν ενεργό μέρος στον εκφοβισμό, μαζί με το θύτη (οι ``βοηθοί``), σ' αυτούς που παρέχουν θετική
ανατροφοδότηση στο θύτη (οι ``ενισχυτές``), σ' αυτούς που μένουν μακριά και
παρακολουθούν από απόσταση το επεισόδιο (οι ``παθητικοί θεατές``) και σ' αυτούς,
που επιχειρούν να παρέμβουν υπέρ του θύματος (οι ``προασπιστές``).

\indent Όπως και να έχει, όταν ο εκφοβισμός συμβαίνει μπροστά σε άλλους και
μένει ατιμώρητος, μπορεί να δημιουργήσει ένα κλίμα φόβου και ανασφάλειας, το οποίο παρεμποδίζει τη μάθηση και αποτελεί μια
σημαντική αιτία περισπασμού, υπονομεύει την απόλαυση του ελεύθερου χρόνου, κάνει
τους μαθητές να αποφεύγουν συγκεκριμένες περιοχές στο σχολείο και δημιουργεί
αυξημένη αίσθηση κινδύνου, στη διαδρομή, από και προς αυτό.

\indent Επιπλέον, οι επιπτώσεις του εκφοβισμού, εκτός από τους συμμετέχοντες,
εκτείνονται συγχρόνως και σε άλλες ομάδες, όπως είναι οι γονείς και τα μέλη της οικογένειας, τα οποία θεωρούνται
δευτερεύοντα θύματα, καθώς βιώνουν τα συναισθήματα λύπης και θυμού του παιδιού
τους, όπως και την αίσθηση της χαμηλής του αυτοεκτίμησης.
\section*{Αιτιολογικοί παράγοντες του σχολικού εκφοβισμού}
\addcontentsline{toc}{section}{Αιτιολογικοί παράγοντες του σχολικού εκφοβισμού}
\indent Οι αιτιολογικοί παράγοντες του σχολικού εκφοβισμού τοποθετούνται στο
επίπεδο \textbf{του ατόμου}, \textbf{της οικογένειας}, \textbf{του σχολείου} και \textbf{της ευρύτερης κοινότητας}.
Δηλαδή, οι παράγοντες κινδύνου σχολικής βίας ακολουθούν ένα ιεραρχικό σχήμα πολλών επιπέδων: μεμονωμένοι μαθητές μέσα σε
τάξεις, τάξεις μέσα σε σχολεία, σχολεία μέσα σε γειτονιές και  γειτονιές μέσα
σε κοινωνίες και πολιτισμούς \ldots

\indent Πολλοί \textbf{ατομικοί παράγοντες} θεωρείται ότι συμβάλλουν στην
εκδήλωση συμπεριφορών σχολικού εκφοβισμού. Μεταξύ αυτών περιλαμβάνονται: η παρορμητική ιδιοσυγκρασία, η θετική στάση προς την
κυριαρχία, η θετική στάση προς την επιθετικότητα, η προηγούμενη εμπειρία
θυματοποίησης από συμπεριφορές εκφοβισμού, η τάση για διασκέδαση του
\mbox{παιδιού - θύτη}, μέσα από τον εκφοβισμό των άλλων
παιδιών - θυμάτων και η έλλειψη \nohyphens{ενσυναίσθησης}.

\indent  Επίσης, συμβάλλουν στη διατήρηση του σχολικού χουλιγκανισμού και
\textbf{οικογενειακοί παράγοντες}: Παραδείγματος χάριν οι θύτες - χούλιγκανς
προέρχονται από οικογένειες, όπου γίνεται συχνή χρήση σωματικής τιμωρίας, όπου οι γονείς είναι εχθρικοί και απορριπτικοί απέναντι στα παιδιά
τους και επιτρεπτικοί προς την επιθετική συμπεριφορά τους, όπου παρατηρείται
έλλειψη γονεϊκής ενασχόλησης, ελέγχου, αλλά και ζεστασιάς και όπου τα παιδιά
``εκπαιδεύονται`` να χρησιμοποιούν τη σωματική επιθετικότητα ελεύθερα, ως ένα
τρόπο  διευθέτησης προβλημάτων μέσα στη σχολική κοινότητα.

\indent Αντιθέτως, τα παιδιά -
θύματα του σχολικού εκφοβισμού από τους συμμαθητές τους, φαίνεται να προέρχονται από
οικογένειες με υπερπροστατευτικούς γονείς. Αυτή, λοιπόν, η γονεϊκή υπερπροστασία
ίσως δυσκολεύει τα παιδιά στο να είναι αποφασιστικά και διεκδικητικά, τα κάνει
να νιώθουν άγχος και ανασφάλεια στις σχέσεις τους με τους συμμαθητές τους και
δεν τους δίνει την ευκαιρία να αναπτύξουν στρατηγικές αντιμετώπισης προβλημάτων
μέσα στην τάξη\ldots

\indent Όμως, εδώ είναι απαραίτητο να τονίσουμε, ότι παιδιά, που κάποτε έπεσαν
θύματα κακοποίησης , κινδυνεύουν και πάλι να γίνουν θύματα σχολικού εκφοβισμού, αλλά υπάρχουν και πιθανότητες επίσης,
να εκδηλώσουν συμπεριφορά θύτη και να εφαρμόσουν στα άλλα παιδιά, ό,τι τα ίδια
υπέστησαν.

\indent Τρίτος αιτιολογικός παράγοντας του σχολικού εκφοβισμού είναι \textbf{το
ίδιο το σχολείο}, δηλαδή το μέγεθός του, οι τάξεις του, η οργανωτική του δομή, οι πολιτικές πρόληψης και αντιμετώπισης του
φαινομένου του εκφοβισμού, ο τρόπος διδασκαλίας, οι σχολικές αξίες, η στάση των
δασκάλων προς τον εκφοβισμό και οι σχέσεις μαθητών - εκπαιδευτικών.

\indent Στο σημείο αυτό, θέλουμε να
τονίσουμε ότι, ο σχολικός εκφοβισμός εντοπίζεται συχνότερα σε σχολεία, που
βρίσκονται σε οικονομικά ασθενέστερες περιοχές, πιθανόν διότι στις περιοχές
αυτές καταγράφονται υψηλότερα επίπεδα αλκοολισμού, χρήσης ναρκωτικών και
βανδαλισμού, υψηλότερα ποσοστά μονογονεϊκών οικογενειών και μια  γενικότερη
εξασθένιση του συνδετικού ιστού της κοινωνίας. Η φτώχεια, οι διακρίσεις, η
αναξιοκρατία, η ανισότητα και η έλλειψη ευκαιριών για μόρφωση και εργασία είναι
παράγοντες κινδύνου διαπροσωπικής βίας και νεανικής επιθετικότητας, οι οποίες
μεταφέρονται στους δρόμους, γύρω από το σχολείο και μέσα σ' αυτό.

\indent Επιπλέον, φαίνεται ότι, τα
μεγάλα σε μέγεθος σχολεία και οι πολυπληθείς τάξεις στερούνται επαρκών εκπαιδευτικών πόρων
και αντιμετωπίζουν περισσότερα προβλήματα στην πειθαρχία και τη διαχείριση της
σχολικής τάξης. Έτσι, έρευνες δείχνουν ότι, στα μεγάλα σε μέγεθος σχολεία
αναφέρονται περισσότερα θύματα σχολικής βίας, αλλά λιγότεροι θύτες, από ό,τι στα
μικρότερα σχολεία και αυτό πιθανόν να οφείλεται στην αδυναμία εντοπισμού των
θυτών, στα μεγάλα σχολεία, αφού η υψηλή αναλογία μαθητών προς εκπαιδευτικούς
έχει αρνητικά εκπαιδευτικά επακόλουθα για όλη τη σχολική κοινότητα. Ένα
αρνητικό σχολικό κλίμα, όπου τη μεγαλύτερη προσοχή αποσπούν οι αντικοινωνικές
συμπεριφορές, διευκολύνει και ενθαρρύνει τα παιδιά να σχηματίζουν κλίκες και να
επιδίδονται στον εκφοβισμό των συμμαθητών τους.

\indent Κάποιες φορές επίσης, ορισμένοι εκπαιδευτικοί, στην προσπάθειά τους να
διατηρήσουν τον έλεγχο της τάξης τους, προβαίνουν - λανθασμένα - σε σωματική και ψυχολογική κακοποίηση των μαθητών
τους, απειλώντας, περιπαίζοντας, ντροπιάζοντας ή φοβερίζοντάς τους, γεγονός, που
αποτελεί για τα παιδιά έναν ακόμη παράγοντα κινδύνου ψυχικής αποξένωσης και
εκδήλωσης εκφοβιστικών ή βίαιων συμπεριφορών από μέρους τους.
\addcontentsline{toc}{section}{Ο κυβερνοεκφοβισμός}
\section*{Ο κυβερνοεκφοβισμός}
\subsection*{Oρισμός του όρου}
\addcontentsline{toc}{subsection}{Oρισμός του όρου}
\indent Μια ακραία μορφή ψυχολογικής βίας στις μέρες μας, είναι ο
κυβερνοεκφοβισμός (CyberBullying), ο οποίος έχει ήδη λάβει ανησυχητικές διαστάσεις για μικρούς και μεγάλους, που χειρίζονται τις σύγχρονες
Τεχνολογίες Πληροφορίας και Επικοινωνιών (κινητό τηλέφωνο, ηλεκτρονικό
ταχυδρομείο, κλπ). Ο όρος αυτός επινοήθηκε από τον Καναδό εκπαιδευτικό Bill
Belsey, δημιουργό του βραβευμένου ιστοτόπου http://www.cyberbullying.org .

\indent Άλλοι όροι, που χρησιμοποιούνται με την ίδια σημασία, είναι:
\textbf{ψηφιακός εκφοβισμός} (Digital bullying), \textbf{ηλεκτρονικός εκφοβισμός} (e-bullying) και \textbf{εκφοβισμός μέσω γραπτών μηνυμάτων} (sms bullying) και αφορούν στην απόδοση του εκφοβισμού στον
Κυβερνοχώρο (Cyberspace), στο περιβάλλον δηλαδή, μέσα στο οποίο η ψηφιακή κοινωνία λειτουργεί και αναπτύσσεται, αλλάζοντας έτσι τη δομή και τη λειτουργία της οικονομίας, την επιχειρηματικότητα και γενικώς την ανάπτυξη και τη
διαχείριση ανθρώπινων πόρων, ενώ άλλαξε ακόμη ριζικά και τον τρόπο, με τον οποίο
τα άτομα διαβιώνουν, εργάζονται, επικοινωνούν και διαθέτουν τον ελεύθερο χρόνο
τους\ldots

\indent Ο όρος ``Κυβερνοχώρος`` χρησιμοποιήθηκε για πρώτη φορά από τον William
Gibson στο βιβλίο του Neuromancer (Νευρομάντης) και αναφέρεται στο σύνολο των ηλεκτρονικών κόσμων, όπως το Διαδίκτυο, όπου οι
άνθρωποι έρχονται σε αλληλεπίδραση, δια μέσου συνδεδεμένων υπολογιστών. Στον
ψηφιακό αυτό κόσμο υπάρχουν ελάχιστοι κανόνες και ουσιαστικά, ο καθένας είναι
ελεύθερος να κάνει ή να λέει οτιδήποτε θέλει, να επινοεί ιστορίες, να διαδίδει
φήμες και να λέει ψέματα, με επιβλαβές για τους άλλους περιεχόμενο, χωρίς
κανένας να μπορεί εύκολα να κάνει κάτι, για όλα αυτά. Έτσι, ο Κυβερνοχώρος
αποτέλεσε ένα πρόσφορο περιβάλλον, για την εκδήλωση παραβατικών και εγκληματικών
συμπεριφορών. Η μεταφορά του εκφοβισμού από το χώρο του σχολείου, υπό την
συμβατική του μορφή, στο χώρο των νέων τεχνολογιών, υπό την νέα ψηφιακή του
μορφή, ήταν απλώς θέμα χρόνου\ldots

\indent Κυβερνοεκφοβισμός - σύμφωνα με το Εθνικό Συμβούλιο Πρόληψης του
Εγκλήματος (National Crime Prevention Council), που είναι ένας μή κερδοσκοπικός εκπαιδευτικός οργανισμός με έδρα την Washington των
Η.Π.Α. -  είναι η σκόπιμη και επαναλαμβανόμενη εχθρική συμπεριφορά, που
εκφραζόμενη μέσω απειλητικών μηνυμάτων και ψευδών εικόνων, δια του διαδικτύου ή
των κινητών τηλεφώνων, αποσκοπεί στην πρόκληση ηθικής βλάβης ή στη  δημιουργία
προβλημάτων και φόβου από ένα άτομο ή ομάδα ατόμων, σε κάποιο άλλο άτομο, με
στόχο τη σπίλωση της υπόληψής του και την καταστροφή της κοινωνικής και
επαγγελματικής του ζωής.
\addcontentsline{toc}{subsection}{Μορφές του κυβερνοεκφοβισμού}
\subsection*{Μορφές του κυβερνοεκφοβισμού}
\indent Οι κύριες μορφές κυβερνοεκφοβισμού - σύμφωνα με την Willard
(\num{2007}), μια αναγνωρισμένη προσωπικότητα σε θέματα ασφαλούς και υπεύθυνης
χρήσης του Διαδικτύου - είναι οι εξής: \textbf{1)} \textbf{Διαδικτυακή
παρακολούθηση} (Cyberstalking) του θύματος. \textbf{2)} \textbf{Σπίλωση και
δυσφήμιση} (Denigration). \textbf{3)} \textbf{Αποκλεισμός ή εξοστρακισμός από το
διαδίκτυο} (Exclusion). \textbf{4)} \textbf{Φλόγισμα} (Flaming), δηλαδή αποστολή
προσβλητικού μηνύματος σε χυδαία γλώσσα. \textbf{5)}
\textbf{Βιντεοσκόπηση με το κινητό} (Happy slapping). \textbf{6)}
\textbf{Διαδικτυακή παρενόχληση} (Harassment). \textbf{7)} \textbf{Υποκλοπή και
χρήση προσωπικού λογαριασμού} (Impersonation ή Masquerade μεταμφίεση).
\textbf{8)} \textbf{Δημοσιοποίηση προσωπικών στοιχείων} (Outing). \textbf{9)}
\textbf{Εμπαιγμός και εξαπάτηση} (Trickery).

\indent Ο Kυβερνοεκφοβισμός, τουλάχιστον στα σχολεία, μέχρι σήμερα, δεν έχει
λάβει ακόμη την έκταση του συμβατικού εκφοβισμού, ωστόσο, οι μελέτες σε Xώρες της Ευρώπης και της Αμερικής
καταδεικνύουν αλματώδη αύξηση ψηφιακής θυματοποίησης, μεταξύ των μαθητών του
Δημοτικού, του Γυμνασίου και λιγότερο του Λυκείου, από συμμαθητές ψηφιακούς
θύτες, ενώ τα ποσοστά των αγοριών και των κοριτσιών είναι περίπου τα ίδια.

\indent Οι συνηθέστεροι τρόποι ηλεκτρονικής επικοινωνίας, για τη διάπραξη του
Kυβερνοεκφοβισμού, είναι: \textbf{α)} \textbf{το ηλεκτρονικό ταχυδρομείο} (e-mail), \textbf{β)} \textbf{τα άμεσα μηνύματα} (Instant Messaging - IM),
\textbf{γ)} \textbf{τα μηνύματα γραπτού κειμένου σε κινητό} (sms), \textbf{δ)}
\textbf{οι φωτογραφίες και τα βίντεο}. 

\indent Ενώ οι διαδικτυακοί τόποι, στους οποίους συνήθως λαμβάνουν χώρα
περιστατικά ψηφιακού εκφοβισμού είναι: \textbf{1)} \textbf{τα εικονικά δωμάτια
συζήτησης} (Chat rooms), \textbf{2)} \textbf{οι ιστότοποι κοινωνικής δικτύωσης}
(Social Networking Websites),  \textbf{3)} \textbf{οι ηλεκτρονικοί πίνακες
ανακοινώσεων} (Message Boards) και \textbf{4)} \textbf{οι προσωπικοί ιστότοποι δημοσκοπήσεων ή ψηφοφοριών} (Personal Polling
Voting Websites).

\section*{Σύγκριση συμβατικού και ψηφιακού εκφοβισμού}
\addcontentsline{toc}{section}{Σύγκριση συμβατικού και ψηφιακού εκφοβισμού}
\indent Μεταξύ συμβατικού και ψηφιακού εκφοβισμού, υπάρχουν εμφανείς ομοιότητες
αλλά και εμφανείς διαφορές. Ο Κυβερνοεκφοβισμός aνήκει στην κατηγορία των έμμεσων μορφών εκφοβισμού, όπου η επαναληπτικότητα
της συμπεριφοράς και η πρόθεση να προκαλέσει βλάβη (κυρίως ψυχολογική και
συναισθηματική) ο θύτης στο θύμα, αποτελούν κοινά γνωρίσματα και του ψηφιακού
και του συμβατικού εκφοβισμού.

\indent Στο συμβατικό εκφοβισμό, τα παιδιά - θύτες χρησιμοποιούν την υπεροχή
τους σε δύναμη και ισχύ, για να επιβληθούν και να αποκτήσουν τον έλεγχο επί των θυμάτων τους. Ο ψηφιακός εκφοβισμός όμως, δεν προϋποθέτει μια τέτοια υπεροχή και οποιoσδήποτε, ακόμη και ανάπηρος, μπορεί να
τον διαπράξει από μακριά, αρκεί να είναι εφοδιασμένος με έναν υπολογιστή ή με
ένα κινητό τηλέφωνο, με τα επιπλέον πλεονεκτήματα, του να μένει άγνωστος και να
μη μπορεί να του επιβληθεί καμία τιμωρία.

\indent Μία άλλη, πολύ σημαντική διαφορά μεταξύ συμβατικού και ψηφιακού
εκφοβισμού είναι, το ότι ο συμβατικός εκφοβισμός παύει, μόλις το θύμα φύγει από το χώρο του σχολείου, ψηφιακό εκφοβισμό όμως,
μπορεί να υποστεί κάποιος, κάθε φορά που συνδέεται στο διαδίκτυο ή κάθε φορά,
που σηκώνει το κινητό του τηλέφωνο και μάλιστα όταν βρίσκεται κυρίως στο σπίτι
του ή οπουδήποτε αλλού και οποιαδήποτε χρονική στιγμή.

\indent Έρευνες έδειξαν ότι, οι θύτες και τα θύματα του ψηφιακού εκφοβισμού
είναι συνήθως θύτες και θύματα και του συμβατικού εκφοβισμού, ενώ κάποια θύματα συμβατικού εκφοβισμού καταφεύγουν, ως
αντιστάθμισμα, στον ψηφιακό εκφοβισμό άλλων. Όμως, η έκταση του φαινομένου αυτού
διαφέρει μεταξύ μαθητών από διαφορετικές χώρες και διαφορετικά πολιτισμικά
περιβάλλοντα. Αυτό αποδίδεται στις διαφορετικές στάσεις, αξίες και πεποιθήσεις,
από τις οποίες εμφορούνται άτομα,  με διαφορετικό πολιτισμικό υπόβαθρο.

\indent Σε κάθε περίπτωση, ο ηλεκτρονικός εκφοβισμός, συγκρινόμενος με τον
συμβατικό, είναι πολύ πιο επικίνδυνος για τα παιδιά κάθε βαθμίδας εκπαίδευσης και αποτελεί την ύψιστη μορφή χουλιγκανισμού
στα σχολεία και στην κοινωνία γενικότερα, επειδή επηρεάζει αρνητικά τη σωματική,
την κοινωνική, τη συναισθηματική και τη γνωστική λειτουργία και ανάπτυξη, καθώς
και το ευ ζήν των παιδιών. Για τον λόγο λοιπόν αυτόν, ο αυξανόμενος αριθμός και
το ανερχόμενο επίπεδο σοβαρότητας των κρουσμάτων Kυβερνοεκφοβισμού σε διεθνές
επίπεδο, έχουν αποτελέσει αντικείμενο ζωηρού ερευνητικού ενδιαφέροντος, κατά τα
τελευταία είκοσι χρόνια, έτσι ώστε σήμερα, πολλά σχολεία να εφαρμόζουν
αποτελεσματικά προγράμματα πρόληψης του επικίνδυνου αυτού φαινομένου, τα οποία
θα αναφέρουμε επιγραμματικά, ως κατακλείδα της παρούσης ανασκόπησής μας, περί
του χουλιγκανισμού στα σχολεία.
\addcontentsline{toc}{section}{Αποτελεσματικές πρακτικές αντιμετώπισης του
σχολικού χουλιγκανισμού}
\section*{Αποτελεσματικές πρακτικές αντιμετώπισης του σχολικού χουλιγκανισμού}
\indent Για την αντιμετώπιση του σχολικού εκφοβισμού απαιτούνται: \textbf{1)}
\textbf{Σχεδιασμός από το σύνολο της σχολικής κοινότητας} (δράσεις και στόχοι). \textbf{2)} \textbf{Διαμόρφωση
πολιτικής}. \textbf{3)} \textbf{Εφαρμογή των στρατηγικών πρόληψης και
παρέμβασης}. \textbf{4)} \textbf{Διατήρηση του αποφασισθέντος προγράμματος
καταπολέμησης της ενδοσχολικής βίας και των βανδαλισμών}.

\indent Για τη δημιουργία κλίματος,που προάγει τις θετικές κοινωνικές σχέσεις μεταξύ των μαθητών, απαιτούνται:
\textbf{A)} \textbf{στρατηγικές πρόληψης} εκ μέρους των διδασκόντων, όπως:
\textbf{α)} \textbf{Η αναστοχαστική πρακτική}  (δηλαδή η ανάλυση και αξιοποίηση
όλων των δράσεων της σύγχρονης παιδαγωγικής, για την αποτροπή πράξεων της μαθητικής
βίας). \textbf{β)} \textbf{Οι διαδραστικές πρακτικές}. \textbf{γ)} \textbf{Η
διάθεση τακτικού χρόνου για τη διαχείριση του προβλήματος} (Circle time).
\textbf{δ)} \textbf{Η συνεργατική μάθηση}. \textbf{ε)} \textbf{Η χρήση της
κοινωνιομετρίας} (για την χαρτογράφηση των σχέσεων των παιδιών στην τάξη).
\textbf{στ)} \textbf{Το παιχνίδι ρόλων}. Και \textbf{ζ)} \textbf{Οι πρακτικές
συμπαράστασης των θυμάτων από τους συμμαθητές τους}.

\indent Για την επίτευξη των αποτελεσματικών πρακτικών, που αναφέραμε, είναι
απαραίτητη η συνεχής καθοδήγηση και επιμόρφωση των εκπαιδευτικών, γιατί - δυστυχώς - οι περισσότεροι από αυτούς
γνωρίζουν ελάχιστα, σχετικά με τον ενδοσχολικό εκφοβισμό και την αντιμετώπισή
του.

\indent \textbf{B)} Στις \textbf{στρατηγικές παρέμβασης}, για την αντιμετώπιση
της σχολικής βίας, εντάσσονται πρακτικές, οι oποίες υποστηρίζουν τους εξής τομείς: \textbf{1)} \textbf{Τα παιδιά που εκφοβίζονται}. \textbf{2)}
\textbf{Τα παιδιά που εκφοβίζουν} (παρέμβαση με όχι τιμωρητική προσέγγιση, ούτε
φόβο τιμωρίας από την Ποινική Δικαιοσύνη, αλλά ούτε και με προσέγγιση των συνεπειών, δηλαδή την αποδοχή από
το θύτη των συνεπειών των πράξεών του, αφού ο σκοπός είναι εκπαιδευτικός και
όχι τιμωρητικός, δηλαδή αποσκοπούμε, ώστε ο θύτης να συμπονέσει το θύμα του).
\textbf{3)} \textbf{Τους συνομηλίκους θεατές}. \textbf{4)} \textbf{Τους
εκπαιδευτικούς}. \textbf{5)} \textbf{Τους γονείς}. \textbf{6)} \textbf{Το
περιβάλλον}.

\indent Μέσω της Βουλής, η Πολιτεία έχει θέσει το νομικό πλαίσιο αντιμετώπισης
της βίας και του χουλιγκανισμού, αλλά - δυστυχώς - αυτό δεν αρκεί, αφού δε φτάνει να έχουμε νόμους, αλλά και να
τους τηρούμε\ldots

\indent Σήμερα, με τα μαζικά μέσα πληροφόρησης και έρευνας που διαθέτουμε,
έχουμε άριστη εκπαίδευση στα σχολεία μας, αλλά δεν έχουμε αληθινή αγωγή και παιδεία, με την έννοια του
αρχαιοελληνικού όρου. Γι' αυτό, είναι ανάγκη η Παιδεία μας, σε όλες τις
βαθμίδες, να γίνει ανθρωποκεντρική και ανθρωπιστική, με ηθικές αξίες και ιδεώδη
και υγιή πρότυπα προς μίμηση. Πρέπει, κοντά στις φυσικές επιστήμες και τα
μαθηματικά, τα παιδιά να εκπαιδευθούν επίσης και πρώτα στην πειθαρχία και στο
σεβασμό προς τον συνάνθρωπο, ώστε σε όλες τις εκφάνσεις της καθημερινότητάς
τους, στην οικογένεια, στο σχολείο, στην κοινωνία και στις διαπροσωπικές τους
σχέσεις, να κυριαρχούν η ευγενής άμιλλα, η υποχωρητικότητα, ο αλτρουισμός, η
συνεργατικότητα, η συλλογικότητα, η εντιμότητα και το ήθος.

\indent Το υλιστικό πνεύμα, η κτητικότητα, ο ατομικισμός και ο εγωισμός οδηγούν
σε κάθε μορφής αντικοινωνικές συμπεριφορές. Ας προσπαθήσει το σχολείο να αμβλύνει αυτή την παραβατικότητα, προσφέροντας στα παιδιά σφαιρική μόρφωση, μέσα από μια άρρηκτη και αρμονική συνεργασία διαλόγου και πράξης, διδασκόντων, γονέων και αρμοδίων περί την Παιδεία φορέων. Τα αποτελέσματα της
μετατόπισης του κέντρου βάρους από το ``είναι`` στο ``έχειν``, τα βλέπουμε -
δυστυχώς - κάθε μέρα, σε όλους τους τομείς της κοινωνικής μας ζωής. Υπάρχει
άμεση ανάγκη το σχολείο να διδάξει στα παιδιά την υπομονή, την ανεκτικότητα, τη
συμφιλίωση, το διάλογο και την ευαισθητοποίηση απέναντι στα γενικότερα
προβλήματα της μαθητικής κοινότητας, αλλά και της κοινωνίας.

\indent Οι γονείς, είναι ανάγκη, να ασχοληθούν με ενδιαφέρον και να σκύψουν
άγρυπνα πάνω από τις συναισθηματικές ελλείψεις των παιδιών τους και να τα αγαπήσουν πραγματικά, προσφέροντάς τους
ζεστασιά, επιβράβευση και ψυχική ασφάλεια. Να τους μάθουν ότι, στη ζωή δεν έχουν
μόνο δικαιώματα, αλλά και υποχρεώσεις και ότι οι κοινωνικοί επαναστάτες δεν
είναι χούλιγκανς, ούτε ρατσιστές, ούτε βάνδαλοι, ούτε πολιτικοποιημένα
``πιόνια`` στα χέρια των εκάστοτε πολιτικών σκοπιμοτήτων.

\indent Γι' αυτό, Πολιτεία, Σχολείο και Οικογένεια ας πάψουν να εθελοτυφλούν
υποκριτικά, μπροστά σε κάθε μορφής βία μικρών και μεγάλων, μαθητών και αθλητικών ή πολιτικών οπαδών και ας φροντίσουν
να αναπτύξουν υγιείς μορφές διασκέδασης, μάθησης και άθλησης, με την ανάδειξη
ουσιαστικά καλλιεργημένων πολιτών, που θα μετέχουν στο κοινωνικό ``γίγνεσθαι``,
διεκδικώντας - με νόμιμο και πολιτισμένο τρόπο - τα δικαιώματά τους.


\begin{thebibliography}{0}
\renewcommand{\headrulewidth}{0pt}
\fancyhead{}
\markboth{}{}{}

\bibitem{SchoolDisch}
Adams T.A., The status of school discipline and
violence, Annals of The American Academy of Politicical and Social
Science, Publication, 2000.
\bibitem{StudentsP}
Agatston P., Kowalski R., Limber S., Students’ perspectives on cyberbullying,
Journal of Adolescent Health, Publication, 2007.
\bibitem{BConditions}
Ah Y., Jeong W. και Cha T., The study on the actual conditions of bullying and
the phychosocial factors affecting bullying behavior, Journal of Fish and Marine
Science of Education, Publication, 2005.
\bibitem{AEbullyingnterv}
Andreou E., Didaskalou E. και Vlachou, A., Evaluating the effectiveness of a
curriculum-based anti-bullying intervention program in Greek primary schools,
Educational Psychology, Publication, 2007.
\bibitem{Manipulate}
Björkqvist K., Lagerspetz K. M., Kaukiainen A., Do girls manipulate and boys
fight? developmental trends in regard to direct and indirect aggression, Journal
of  Aggressive Behavior, Publication, 1992.
\bibitem{LinksSocial}
Camodeca M., Goossens F. A., Schuengel C., Meerum Terwogt, M., Links
between social information processing in middle childhood and involvement in
bullying, Aggressive Behavior, Publication, 29, 116-127, 2003.
\bibitem{Cyberguise}
Campbell, Marilyn A.,  Cyber bullying: An old problem in a new guise ?,
Australian Journal of Guidance and Counselling, Publication,  2005.
\bibitem{MiddleReports}
Casey-Cannon S., Hayward C., Gowen K., Middle school girls' reports of peef 
victimization: Concerns, consequences and implications.Professional School
Counceling,  The Journal Of Early Adolescence, Publication, 2001.
\bibitem{Marital}
Davies Patrick T., Cummings E. M., Marital conflict and child adjustment: An
emotional security hypothesis, Psychological Bulletin, Publication, 1994.
\bibitem{CVCrime}
Eitle D., Turner R.J., Exposure to CV and young adult crime: The effects of
 witnessing violence, traumatic victimization, and other stresful life events, Journal of
 Research in Crime and Delinquency, Publication, 2002.
 \bibitem{KenRigby}
Ken Rigby, Bullying in Schools and What to Do about It: Revised and Updated,
Aust Council for Ed Research, Publication, 2007.
\bibitem{AtlasBullyingClassroom}
Rona S.,  Atlas and Debra, J. Pepler, Observations of Bullying in the Classroom,
The Journal of Educational Research, Publication, 1998.
\bibitem{CyberBullying}
Slonje R., Smith P.K., Cyberbullying: another main type of bullying ?,
Scandinavian Journal of Phychology, Publication, 2008.
\bibitem{Nbullying}
Smith P. K., Morita Y., Junger-Tas J., Olweus D., Catalano R., και  Slee
P., The Nature of School Bullying: A Cross-national Perspective, Routledge
Publication, Andover, U.K., 1998.
\bibitem{OnlineBullying}
 Wolak J., Mitchell K.J., Finkelhor D., Does online harassment constitute
 bullying? An exploration of online harassment by known peers and online-only
 contacts, Journal of the Adolescent Health, Publication, 2007.
% \bibitem{AEbullying812}
%  Andreou E. Bully/victim Problems and their association with
% Psychological Constructs in 8 to 12 year-olds, εκδ. Aggressive Behaviour, 2000.
% \bibitem{AEbullyingPeers}
% Andreou E., Bully/victim Problems and their association with coping
% behavior in conflictual peer interactions among school-age children, εκδ.
% Educational Psychology, 2001.
\bibitem{AEViolence}
Ανδρέου Ε., Η βία στο σχολείο ως συλλογική διεργασία: Ψυχοκοινωνικές στάσεις
και αλληλεπιδράσεις μαθητών και μαθητριών που εμπλέκονται σε περιστατικά
θυματοποίησης, εκδ. Ελληνικά Γράμματα, Αθήνα, 2004.
\bibitem{ViolenceTech}
Γκουντσίδου Β., Τσιλιγκίρογλου-Φαχαντίδου Α., Βία και νέες τεχνολογίες,
εκδ. Αθλητιατρική, 3 (Συμπληρωματικό τεύχος 1), 25, 2008.
\bibitem{SexIntentity}
Δεληγιάννη-Κουμτζή Β., Ταυτότητες φύλου, εθνικές ταυτότητες, και σχολική βία:
Ερευνώντας τη βία και τη θυματοποίηση στο σχολικό χώρο. Ενδιάμεση έκθεση του
προγράμματος Πυθαγόρας. Περίοδος 1/3/2004 - 31/3/2005, 2005.
\bibitem{HyriakidisBullyingClassroom}
Κυριακίδης Σ.Π., Θύτες και θύματα: ερευνητικά δεδομένα και τρόποι αντιμετώπισης
της θυματοποίησης στο σχολείο, εκδ. Επιθεώρηση Κοινωνικών Ερευνών, 2007.
\bibitem{ViolenceSchools}
Πετρόπουλος Ν., Παπαστυλιανού Α., Μορφές επιθετικότητας, βίας και διαμαρτυρίας
στο σχολείο, εκδ. Παιδαγωγικό Ινστιτούτο, Αθήνα, 2001.
\end{thebibliography}
\pagebreak\centering\setlength{\parindent}{10pt} 
\addcontentsline{toc}{chapter}{Περιεχόμενα}
%\renewcommand{\contentsname}{Περιεχόμενα}
\tableofcontents
\end{document}
